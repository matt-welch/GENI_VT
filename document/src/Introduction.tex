\chapter{Introduction}
\label{sec:introduction}
Virtualization has become the \emph{de facto} platform for supporting flexible, dynamic, and performant server environments.  
Its uses are widespread and present in systems from embedded systems where each virtual machine (VM) may be responsible for a single subsystem, to massive data centers where VMs process network data at tens of gigabits per second.
Interest in these systems has inspired innovation and development in technologies to improve their performance.
Recent innovations in operating system process isolation, specifically for Linux, has inspired the growth and rapid development of process containers.
The very popular Docker container system \autocite{dockerdotcom}, has fueled a renewed interest in their uses in the enterprise.

\section{Use-Cases for Virtualization} % (fold)
\label{sec:introusecasesvt}
The use cases for virtualization are numerous and growing. 
Recently, the ``cloud'' and cloud services have been the subject of significant interest and development.
Growth has been strong in cloud services for good reason \autocite{_younge_1}.
In that work, Younge, et al. describe many of the benefits offered by cloud services such as scalability, controlling quality of service (QoS), customization of user infrastructure, cost effectiveness, and simplified access interfaces, almost all of which are enabled by virtualization.  
Cloud infrastructure commonly requires large-scale data centers with high speed networking to flexibly support large numbers of systems.  
High Performance Computing (HPC) shares many of the infrastructure requirements and capabilities of these large computing centers so HPC may also benefit from innovations originating in the cloud and virtual systems \autocite{xavier2013performance, _younge_1}.

Virtual machines have been the common case for systems virtualization with containers historically taking a secondary role as a process isolation mechanism.
Network Functions Virtualization (NFV) has been an agent for growth recently, driving performance improvements in both virtual machines and containers.
In NFV, the functionality that is often embodied in physical devices such as routers, firewalls, and load balancers may be accomplished with a virtual system which increases flexibility and scalability of the network in many cases.
In their survey of network virtualization, Chowdhury and Boutaba describe some of the older technologies and challenges that define the NFV problem space~\autocite{_chowdhury_1}.
Innovations in containers and their supporting systems in Linux should enable new uses cases in NFV and large-scale systems.
In anticipation of this important transformation of the industry, trade organizations have recently sprouted up to promote NFV adoption and accelerate its development.
Both the European Telecommunications Standards Institute (ETSI) and Open Platform for Network Functions Virtualization (OPNFV) have been formed for that purpose \autocite{opnfv1, cohnopnfv}.

Aside from growth in NFV, cloud computing, and HPC, the need for high-throughput systems with deterministic responses is also expanding.
The Linux Foundation recently announced that they are fully supporting the efforts to promote and advance the development of realtime Linux \autocite{_linux_foundation_1}.
The Linux Foundation is the body that guides and sponsors the direction of Linux so their interest comes at an important time due to recent funding difficulties for realtime Linux \autocite{_lwn_1}.
Section~\ref{sec:results} shows that system determinism and responsiveness of virtual systems can also benefit from developments to realtime Linux.

Although virtualization is a mature technology with decades of enterprise use, it is still being improved by constant innovation.  
Many of the important developments in virtualization are presented here with a discussion of their importance in order to better understand the motivation for various configuration choices made in Section~\ref{sec:experimental_design}.  

% section introusecasesvt (end)

\section{Performance Analysis of Virtual Systems} % (fold)
\label{sec:introperformanceanalysis}
The use cases for virtualization in network-resident workloads are numerous, but performance of virtual systems is often a concern.
As Section~\ref{sub:performanceanalysis} demonstrates, performance analyses of containers and virtual machines seldom address modifications to host or hypervisor systems to increase their isolation from the rest of the system or make improvements on default settings.  
In this thesis, preempt-rt patches are applied to a Linux kernel and the kernel configuration is modified in order to improve system determinism and responsiveness. 
In order to improve the performance of virtual systems, the host system's kernel may be modified and booted with parameters that enhance the determinism of the system as a whole and guest I/O latency and jitter in particular.
The intent of this tuning is to demonstrate the significant improvements that can be realized in network performance of virtual systems with straightforward modifications to operating system configuration.

Architectural details of virtual machines and containers are presented in Sections~\ref{sec:vmarchitecture} and \ref{sec:containerarchitecture}, respectively to outline opportunities for performance improvement.  
Innovations to virtual machine and container architecture are also discussed with an emphasis on their relevance to performance.
The methods of kernel tuning and system optimizations that enable enhanced performance in virtual systems are discussed in \ref{sec:experimental_setup}.
% section introperformanceanalysis (end)

This thesis is divided into two parts.
Part I, composed of Chapters~\ref{sec:related_work}, and \ref{sec:experimental_design}, describes some of the concerns involved in virtual systems and the work that has been done to improve their performance.
Part II, comprising Chapters~\ref{sec:results} and \ref{sec:conclusions}, describes the results of the system tuning and performance comparisons completed here, conclusions that may be drawn from them, and the potential for further analysis.
