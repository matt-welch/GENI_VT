Virtual machines and containers have steadily improved their performance over time as a result of innovations in their architecture and software ecosystems.
Network functions and workloads are increasingly migrating to virtual environments, supported by developments in software defined networking (SDN) and network functions virtualization (NFV).
Previous performance analyses of virtual systems in this context often ignore significant performance gains that can be acheived with practical modifications to hypervisor and host systems.
In this thesis, the network performance of containers and virtual machines are measured with standard network performance tools.
The performance of these systems utilizing a standard 3.18.20 Linux kernel is compared to that of a realtime-tuned variant of the same kernel.  
This thesis motivates improving determinism in virtual systems with modifications to host and guest kernels and thoughtful process isolation. 
With the system modifications described here, the median TCP bandwidth of KVM virtual machines over bridged network interfaces, is increased by 10.8\% with a corresponding reduction in standard deviation of 87.6\%.
Docker containers see a 8.8\% improvement in median bandwidth and 4.4\% reduction in standard deviation of TCP measurements using similar bridged networking.
System tuning also reduces the standard deviation of TCP request/response latency (TCP RR) over bridged interfaces by 86.8\% for virtual machines and 97.9\% for containers.
Hardware devices assigned to virtual systems also see reductions in variance, although not as noteworthy.


