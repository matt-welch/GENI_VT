\glsresetall
\chapter{Related Work}
\label{sec:related_work}
Virtualization has been an important part of computing as soon as more than one user wanted to share the 
expensive mainframe computer.  This section will describe some of the work in the field of 
virtualization that has brought us to the current state of the art.  We will also discuss some of the
subsystems in Linux and processor hardware that enable virtualization. 

\section{History of Virtualization}
\label{sec:history_vt}
Virtualization has been an important component in computing since the invention of virtual memory to support multiprogramming on mainframes in the 60's.  
Without going into too much detail:  
\begin{enumerate}
  \item Virtualization in early mainframes (IBM 360)
  \item Early commercial virtualization
  \item Entrance of Virtualization in x86
  \item Virtualization as growth driver for datacenters and enterprise networks
  \item Containers
  \item Desktop virtualization
  \item NFV
\end{enumerate}

"The Linux Foundation Announces Project to Advance Real-Time Linux"  [linux\_foundation\_2015]
	the Linux Foundation is the body that guides and sponsors the direction of Linux.  
	Realtime Linux has been struggling to find funding until recently [lwn.net reference here] 
	This project is important because realtime Linux will also improve virtualization determinism.
	this will ensure that the realtime branch of Linux is supported going forward.  


\section{Architecture}
\label{sec:sys_arch}
Virtualization has been around in many forms.  Virtual machines and containers, specifically Docker, are currently 
very popular so we will discuss the architecture of these types of virtualization here.
\subsection{Virtual Machines}
\label{sec:sys_arch_vm}
The architectures available in virtual environments are as widely varied as the types 
of hardware they seek to emulate.  Many of the concepts we use regarding virtual machines,
however, originated in a seminal 1974 paper by Popek and Goldberg \autocite{popek_1}.  Although 
virtual machines had already been implemented on "third-generation computer systems" 
such as the IBM 360/67, Popek and Goldberg sought to establish formal requirements 
for and prove whether other systems of that era were capable of supporting a 
"virtual machine monitor" \autocite{popek_1}.  At the time of that writing, their analysis 
was on the possibility of a Virtual Machine Monitor (VMM), but the term hypervisor
has largely come to replace VMM as the name of a software system that allocates,
monitors, and controls virtual machine resources as an intermediary between the hardware
and the virtual machine's operating system.  What follows in this section is an 
illustration of some important concepts in hypervisor and virtual machine architecture
and the work that led to them.

\subsection{Type 1 vs. Type 2 Hypervisors}
\label{sec:sys_arch_vm_type}

