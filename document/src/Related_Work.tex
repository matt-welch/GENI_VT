\glsresetall
\chapter{Related Work}
\label{sec:related_work}
Virtualization has been an important part of computing as soon as more than one user wanted to share the 
expensive mainframe computer.  This section will describe some of the work in the field of 
virtualization that has brought us to the current state of the art.  We will also discuss some of the
subsystems in Linux and processor hardware that enable virtualization. 

\section{History of Virtualization}
\label{sec:history_vt}
Virtualization has been an important component in computing since the invention of virtual memory to support multiprogramming on mainframes in the 60's.  
Without going into too much detail:  
\begin{enumerate}
  \item Virtualization in early mainframes (IBM 360)
  \item Early commercial virtualization
  \item Entrance of Virtualization in x86
  \item Virtualization as growth driver for datacenters and enterprise networks
  \item Containers
  \item Desktop virtualization
  \item NFV
\end{enumerate}

"The Linux Foundation Announces Project to Advance Real-Time Linux" \autocite{_linux_foundation_1}
\begin{enumerate}
  \item the Linux Foundation is the body that guides and sponsors the direction of Linux.  
  \item Realtime Linux has been struggling to find funding until recently [lwn.net reference here] 
  \item This project is important because realtime Linux will also improve virtualization determinism.
  \item this will ensure that the realtime branch of Linux is supported going forward.  
\end{enumerate}


\section{Virtual Machine Architecture}
\label{sec:vm_arch}
Virtualization has been around in many forms.  Virtual machines and containers, specifically Docker, are currently 
very popular so we will discuss the architecture of these types of virtualization here.
\subsection{Virtual Machines}
\label{sec:virtual_machines}
The architectures available in virtual environments are as widely varied as the types 
of hardware they seek to emulate.  Many of the concepts we use regarding virtual machines,
however, originated in a seminal 1974 paper by Popek and Goldberg \autocite{_popek_1}.  Although 
virtual machines had already been implemented on "third-generation computer systems" 
such as the IBM 360/67, Popek and Goldberg sought to establish formal requirements 
for and prove whether other systems of that era were capable of supporting a 
"virtual machine monitor" \autocite{_popek_1}.  At the time of that writing, their analysis 
was on the possibility of a Virtual Machine Monitor (VMM), but the term hypervisor
has largely come to replace VMM as the name of a software system that allocates,
monitors, and controls virtual machine resources as an intermediary between the hardware
and the virtual machine's operating system.  What follows in this section is an 
illustration of some important concepts in hypervisor and virtual machine architecture
and the work that led to them.

\subsection{Hypervisors}
\label{sec:hypervisors}
Hypervisors come in a few common flavors, depending on the ["level of abstraction" or "level of virtualization"] 
where the hypervisor sits in relation to the hardware and the guest operating system.
In a Type1 hypervisor, the hypervisor is running directly on the CPU or "bare metal".
That is, the hypervisor code is not being hosted, translated, or controlled by another
system or piece of software, but running as a native process on the CPU.  Type 2
hypervisors, however, require a host operating system to provide some system services
including memory and filesystem management.

The ideal difference between these two types of hypervisors is that the Type 1 hypervisor does not require an operating system to run, whereas the Type 2 does.  The truth is that, although Type 1 hypervisors do not require an operating system to RUN, they DO require an operating system for CONTROL \autocite{_liguori_1}.  It would appear that Type 1 systems seem to have an efficiency advantage over type 2 since they tend to use microkernels instead of the macrokernels that are usually required to host Type 2 hypervisors.  According to \autocite{_liguori_1}, however, the difference between them has little to do with performance, robustness, or other qualitative factors, but, rather, relates back to  observations about their differences made in \autocite{_popek_1} and the analysis of these differences by \autocite{_robin_1}.  Robin's analysis related to the potential for the Pentium processor to support a secure HVM \autocite{_robin_1}.  

Although it's arguable whether there is a performance difference between Type 1 and Type 2 hypervisors, the market for virtualization is full of both.  VMware, one of the more popular enterprise virtualization vendors, has products covering both architectures including their Type 1 ESX enterprise product that runs on large-scale virtualized systems over the world \autocite{_vmware_0}.  They also provide Type 2 hypervisors such as VMware Workstation Player that runs on systems from desktops to small-scale servers that don't need the higher levels of orchestration provided by ESX.  

In addition to numerous other available hypervisors, the Kernel Virtual Machine (KVM) is the de facto hypervisor in Linux that has been included as a module in the Linux kernel since released February 5, 2007 when it was included in Linux kernel version 2.6.20 \autocite{_kvm_1}.  The KVM module allows the host operating system to provide the low-level hardware access to the guest that has been enabled by hardware virtualization extensions such as Intel's VT-x.  [elaborate more on KVM here and why we're using it].  Along with KVM in Linux, there is a user-space component known as the Quick EMUlator or QEMU\autocite{_qemu_1}.  

The following paragraph is a direct quote and should be rewritten: QEMU is a generic and open source machine emulator and virtualizer.  When used as a machine emulator, QEMU can run OSes and programs made for one machine (e.g. an ARM board) on a different machine (e.g. your own PC). By using dynamic translation, it achieves very good performance.  When used as a virtualizer, QEMU achieves near native performances by executing the guest code directly on the host CPU. QEMU supports virtualization when executing under the Xen hypervisor or using the KVM kernel module in Linux. When using KVM, QEMU can virtualize x86, server and embedded PowerPC, and S390 guests. \autocite{_qemu_1}.  

Combined, the KVM/QEMU hypervisor runs natively on Linux with kernel modules and userspace tools, but additional layers such as libvirt \autocite{_libvirt_1} can be utilized to simplify VM creation, monitoring, and orchestration.  In this work, we have chosen to eschew any higher-level libraries such as libvirt or VMware orchestration in favor of running as efficiently as possibly without additional software layers of abstraction slowing things down.  The hope is for maximal performance so as few layers as possible were included in the system configuration.


\subsection{CPU Virtualization}
\label{sec:vt_cpu}
In addition to the variations in hypervisors that may or may not have an impact on performance, the 
type of virtualizion used to provide devices to the guest operating system can have a significant impact
on performance.  Following, we will review the various types of hardware emulation comprising virtio, 
paravirtualization, emulation, binray translation, and hardware assisted.

One of the early challenges in the virtualization of the x86 architecture was how privileged instructions should be handled.  The x86 processor was designed with 4 "rings", numbered from 0 to 3, representing decreasing privilege levels as the rings increase [x86 ring architecture reference here].   Operating systems utilizing the x86 instruction set execute user code in ring 3 and privileged instructions (kernel code) in ring 0.  Virtual machines on x86 execute their instructions as a user-space process in ring 3 so the host OS can maintain control of the system.  How the processor executes privileged instructions on behalf of the guest has been the topic of considerable effort in the advancement of virtualization and has spawned multiple techniques for handling these instructions. 

One of the earliest methods developed by VMware, known as binary translation involved trapping privileged guest instructions in the hypervisor and translating them into "sequences of instructions that have the intended effect on the virtual hardware" \autocite{_vmware_1}.  The binary translation technique was one of the most efficient methods in early hypervisors, but is now considered to have high overhead due to the additional time required to perform the code translation.  This same translation process, however, allows a hypervisor using this technique to run guest operating systems for virtually any processor on any other instruction set, provided that an efficient translation can be achieved. 

Paravirtualization is another technique for handling guest privileged instructions.  It involves cooperation between the guest and host operating systems to improve efficiency.  The guest OS must be modified to replace privileged instructions "with hypercalls that communicate directly with the virtualization layer hypervisor." \autocite{_vmware_1}.  VMware has incorporated this method in their vmxnet series of network drivers to accelerate network workloads.  A more widely known example of paravirtualization is one of the first open-source hypervisors, known as the Xen hypervisor.  At the risk of oversimilification, the Xen project is, essentially, a modified Linux host that communicates directly with the guest kernels.  The guest kernels and modules must also be modified to facilitate this communication which places an additional burden on hardware vendors to provide not only open source drivers but also paravirtualized open source drivers for their hardware.  Although the Xen hypervisor was originally developed with the intent of being hardware agnostic \autocite{_barham_1}, the modifications required to the guest operating system mean that only vendors wishing to participate in the open-source community will provide paravirtualized drivers.  The community itself is free to develop these drivers, but this adds a barrier to adoption for new hardware products.  Despite the additional effort required, Xen maintained their lead as a very popular open-source hypervisor for many years.  Paravirtulaizaion has its fans, however, and much work has been done previously comparing Xen \autocite{_felter_1, _younge_1, _wang_1, _che_1, _scheepers_1, _wang_2}.

The third popular virtualization method we will discuss is what appears to be the de-facto standard as virtualization matures.  Hardware Assisted virtualization, which is heavily promoted by CPU vendors such as Intel and AMD, started out as an effort to accelerate instruction translation, but early generations of hardware had difficulty keeping up with the binary translation preferred by VMware \autocite{_vmware_1}.  The silicon vendors, however have been hard at work improving their VT performance.  "Hardware assist introduced a new CPU execution mode that allowed the hypervisor to run below ring 0.  Privileged instructions that would normally be trapped are set to automatically trap to the hypervisor, removing the need for binary translation or paravirtualization" \autocite{_vmware_1}.  For each virtual machine, a new structure called a Virtual Machine Control Structure (VMCS-IA) or VM Control Block, (VMCB-AMD), which is logically similar to a process control block,  is maintained in the host kernel memory. \autocite{_vmware_1} These structures maintain the state of the virtual machine and are updated when the guest needs to perform a "vm-exit" to allow the host to execute privileged instructions.  This vm-exit process was initially a significant source of latency for any guest privileged instructions, but advancements in virtualization-aware driver models like SR-IOV [reference here??] have improved both latency and the frequency of these exits.  

Although a deep discussion could be had regarding the hardware systems and mechanisms of processor and platform virtualization, that discussion is slightly outside the scope of this document.  Suffice to say, however, that the CPU vendors have significant interest in producing higher core-count CPUs and supporting virtualization.  We limit the discussion of these processors even further to x86 architecture, but it should be noted that ARM and other vendors are actively working to enable virtualization with similar methods.  Additionally, important components of virtualization such as SR-IOV are managed by the PCI-SIG and are not the exclusive domain of x86.  To that end, both Intel and AMD have independently developed processor extensions to enable virtualization.  [ more explanation of what they've done here ]  

\subsection{Memory Virtualization}
\label{sec:vt_memory}
Virtualization of the CPU was the first challenge in the development of secure virtualization, but, in Von Neumann processor architecture, the memory unit is equally important.  Memory has become an increasingly  significant source of latency in processors as the CPU core has improved its performance faster than the improvement of memory so improvements in this area are doubly beneficial for virtual machines.  The host system must carefully control access to memory which holds the system state along with program data.  Virtualization uses memory structures similar to those developed for virtual memory in early computing systems.  Virtual memory was created to allow multiprogramming and process address spaces that are larger than the available physical memory.  Modern CPUs implement virtual memory with the aid of a memory management unit (MMU) and translation lookaside buffer (TLB) to manage page tables and accelerate page lookups.  Fairly recent developments in x86 processors have allowed hypervisors to maintain guest memory mappings with shadow page tables, known as Extended Page Tables (EPT) in Intel processors and Nested Page Tables (NPT) for AMD [reference??].  A hypervisor may use TLB hardware to map guest memory pages onto the physical memory similarly to a native process, reducing the overhead of guest OS memory access.  
Along with virtualization of guest memory, another significant improvement to virtual machine performance can be had with the utilization of hugepage memory \autocite{_romer_1}.  Hugepages (superpages in Romer), can simply be described as a memory page that is a power-of-two multiple of a standard 4096 byte (4k) memory page.  Romer et. al. demonstrated a significant improvement in performance when using hugepages for memory-hungry applications.  When system memory sizes were very small, virtual memory and swapping smaller pages was found to be more efficient than larger pages.  As memory sizes have grown to hundreds of gigabytes per server, 4k memory pages do not seem like such an obvious fit.  Hugepages seek to reduce the frequency of TLB lookups and page table walks by simply using larger pages of memory.  Romer showed that they could reduce TLB overhead by as much as 99\% using \emph{superpage promotion} which is a system of aggregating smaller pages together as they are used \autocite{_romer_1}.  Current implementations of hugepages in Linux, however, offer a set of large memory pages available to the kernel for memory allocation [reference here].  Unlike Romer's superpages, hugepages must be allocated at boot, rather than being coalesced dynamically, but the performance of the Linux hugepages are similar except for increased memory consumption due to unused portions of hugepages.  
For virtual machines that are potentially using multiple levels of memory page walk, thereby increasing the latency of those operations, any reduction in the frequency of lookups will be beneficial.  Hugepages are not standard practice, however, so we have chosen to include hugepages as a primary mechanism for improving the memory performance of our virtual machines in our tuned systems.  
One popular use of hugepages in accelerating network workloads is through the Data Plane Development Kit, first introduced by Intel and now an open source project \autocite{_dpdk_1}.  [I would really like to use DPDK, but don't think I have time so maybe I should just drop it here :( ]

\subsection{I/O Virtualization}
\label{sec:vt_io}
After achieving performant virtualization solutions for compute and memory, the last component in achieving full system virtualization was the virtualization of devices and I/O (IOV) [ history reference?].   Along with compute and storage, processing of I/O is one of the most critical components of a server's workload [because??].  Efficiently utilizing I/O allows virtual machines to perform nearly any task possible in a native system, particularly the processing of network packets and workloads.  Similar to the "virtualization penalty" that occurs with nested page table lookups and binary translation, latency inherent in the processing of network packets with multiple levels of handoffs should be avoided when possible.  
Virtualization is essentially software emulation of hardware devices so the natural first attempt at device virtualization is to emulate a device in the kernel, providing a software device to the guest OS that is based on a common physical device.  This method is common in some workstation virtualization products such as VMware workstation \autocite{_jones_1}.  Instead of emulating a software device in the kernel, it is also possible to emulate the device in user-space.  This is the method used by QEMU which provides both device emulation and a platform-agnostic hypervisor.  In Linux operating systems, QEMU is often combined with the KVM modules with QEMU providing user-space device emulation for simple devices such as mouse and keyboard and KVM providing virtualization of the physical hardware.  Userspace emulation has the advantage of removing the responsibility from the kernel, thereby minimizing the potential attack surface of the kernel.  As mentioned earlier, paravirtualized devices are another variation on this theme of emulation where the paravirtualized guest driver communicates with the host paravirtualized devices.  In addition to Xen's paravirtualized driver model, the KVM virtio library is the basis for paravirtualized devices in Linux [virtio1].  The virtio library takes inspiration from paravirtualization and userspace emulation and uses qemu to implement the device emulation in userspace so host drivers are not necessary [virtio1].      
While device emulation can provide important flexibility and hardware independence, it brings up the recurring theme of software managing hardware functions which is often less efficient than utilizing dedicated, specialized hardware to perform the function.  The alternative is to avoid any device emulation and allow the guest OS to access hardware directly as if it belonged to the guest rather than the host system.   Since it is conceptually similar to the MMU virtualization for memory, virtualizing the DMA transactions of modern I/O devices requires an I/O MMU (IOMMU) to allow communication between the guest and I/O devices.  In x86 architecture, this feature is known as AMD-Vi or Intel VT-d, but both utilize the same concept of an IOMMU to avoid vm-exits when processing I/O.  This allows the hypervisor to remove a hardware device from itself and "pass through" or "direct assign" the device to the guest OS \autocite{_jones_1}.  Direct assignment of hardware to guests also comes with the high cost of dedicating network interfaces or other important devices to a guest, but these devices provide significant performance improvements over paravirtualized or emulated guest devices, thereby enabling applications that could not previously be virtualized due to low-latency constraints \autocite{_jones_1}.  
If a host system has a large number of virtual machines that need high-performing I/O, it can be difficult to fit enough peripheral cards into the host chassis to passthrough devices to the virtual machines, thus providing each VM with dedicated hardware.  A solution to this apparent conflict of interests can be found in "PCI-SIG Single Root I/O Virtualization" (SR-IOV) [pcisig1].  SR-IOV is a general method by which the host system can configure a single hardware device to create and control virtual functions of the device.  These virtual functions can be directly accessed by the guest, similar to passthrough, without removing the main physical function from the host.  An illustrative example is the Intel 82599 10 gigabit network card, used later in this study.  Each physical network interface (physical function or pf) in these network cards has 64 Rx/Tx queue pairs that are normally used as receive and send buffers to maintain multiple simultaneous flows.  The individual queue pairs may be "broken out" of the main pool to create a virtual function that is, essentially, a new network device sharing the same physical interface as the physical function.  These new devices are assigned unique MAC addresses and IP addresses to differentiate them on the network so that an outside observer cannot tell them apart from a physical device.  The advantage here is that one interface can be multiplexed into multiple independent interfaces that can each reach near-native performance levels [nasa1].  As we will see later, this method has great potential in systems supporting large numbers of virtual machines or containers with only limited host physical resources.  


\subsection{Hypervisor Options}
\label{sec:hypervisor_options}

\section{Container Architecture}
\label{sec:container_arch}

\subsection{Operating System Virtualiztion}
\label{sec:os_vt} 

\subsection{Container Systems}
\label{sec:container_systems}

\subsection{Resource Scope}
\label{sec:resource_scope} 

\subsection{Resource Control}
\label{sec:resource_control}



\nocite{_adams_1, _chowdhury_1, _perry_1, _grinberg_1} 

