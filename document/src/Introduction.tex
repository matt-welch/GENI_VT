\chapter{Introduction}
\label{cha:introduction}
\label{sec:introduction}
Virtualization has become more than an academic exercise to get multiple simultaneous operating systems running on a single box.  
It has become the \emph{de facto} paradigm for supporting flexible, dynamic, and performant servers in systems from automotive where each VM is responsible for a single subsystem to massive datacenters where virtual machines process gigabits per second of network data.

Recent innovation in operating systems, specifically Linux, has inspired the growth and rapid development of process containers.

Most performance comparisons of containers and virtual machines do not address the idea that these systems can be adjusted to fit various roles, depending on the level of resources avilable to them.  


This thesis is divided into two parts.
Part one, composed of Chapters~\ref{sec:related_work}, and \ref{sec:experimental_design} describes some of the concerns involved in virtual systems and the work that has been done to improve their performance.
Part two, comprising Chapters!\ref{sec:results} and \ref{sec:conclusions} describe the performance results of the comparison performed herein and the potential for further analysis.
