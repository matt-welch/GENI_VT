\chapter{Conclusions \& Future Work}
\label{cha:conclusions}
\label{sec:conclusions}
Conclusions text here
Although it may be a controversial statement, it is the author's opinion that it is somewhat incorrect to think of containers as \emph{virtualization} per se.  
Without arguing the definition of virtualization, a container does little to actually virtualize anything, but is more of a system for isolating and limiting the resource consumption of processes while providing dependencies necessary for them to run.  
The only actual virtualization that is happening in a container is the masking of resources and, perhaps virtualization of network or other I/O resources to enable controlled sharing with other containers and processes in the system. 


\section{Future Work}
\label{sec:future_work}
\subsection{DPDK}
DPDK is, like, super awesome and should be used by everyone!
Optimizing parameters of the network stack in Linux or implementing benchmarks in DPDK are additional optimizations for improving network performance and may be considered in future work.

\subsection{Virtual Functions}
\label{sec:virtual_functions}
Virtual functions would be really good for scaling

