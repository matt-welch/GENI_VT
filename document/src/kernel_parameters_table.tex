\begin{table}
    \centering
    \caption{Kernel Boot Parameters and Their Descriptions}
    \label{tab:kernel_params_table}
\begin{tabular}{|l|p{10cm}|}
    \hline
    Parameter & Description \\
    \hline \hline
    \texttt{isolcpus=1-4} & Isolate CPUs 1-4, removing them from the scheduler pool of available CPUs. \\ 
    \hline
    \texttt{hugepagesz=2M} & Set hugepage size to 2 MB (from default 4 KB). \\
    \hline
    \texttt{hugepages=4096} & Allocate 4096 hugepages for a total of 8192 MB. \\ 
    \hline
    \texttt{nohz\_full=1-4} & Set scheduler timer interrupt interval to 1 Hz instead of the 250 Hz default. \\ 
    \hline
    \texttt{rcu\_nocbs=1-4} & Set the specified list of CPUs to be no-callback CPUs. This reduces OS jitter on the offloaded CPUs, which can be useful for HPC and realtime workloads. It can also improve energy efficiency for asymmetric multiprocessors. \\ 
    \hline
    \texttt{rcu\_nocb\_poll=1} & Rather than requiring that offloaded CPUs (specified by \texttt{rcu\_nocbs=} above) explicitly awaken the corresponding \texttt{rcuoN} kthreads, make these kernel threads poll for callbacks. This improves the realtime response for the offloaded CPUs by relieving them of the need to wake up the corresponding kernel thread, but degrades energy efficiency by requiring that the kernel threads periodically wake up to do the polling. \\
    \hline
\end{tabular}
\end{table}
