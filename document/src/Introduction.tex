\chapter{Introduction}
\label{sec:introduction}

Virtualization technologies enable computing paradigms that have become the \emph{de facto} platform for supporting flexible, dynamic, and performant server environments.  
Their uses are widespread supporting systems from automotive, where each virtual machine (VM) is responsible for a single subsystem, to massive data centers where virtual machines collectively process hundreds off gigabits per second of network data.
Interest in these systems has inspired innovation and development in technologies to improve their performance.
Recent innovations in operating system process isolation, specifically for Linux, has inspired the growth and rapid development of process containers.
The very popular Docker container system \autocite{dockerdotcom}, has fueled a resurgence in their uses in the enterprise.

\section{Use-Cases for Virtualization} % (fold)
\label{sec:introusecasesvt}
The use cases for virtualization are numerous and growing. 
Recently, the ``cloud'' and cloud services have been the subject of significant development and interest.
Growth has been strong in cloud services for good reason \autocite{_younge_1}.
They describe many of the benefits offered by cloud services such as scalability, controlling quality of service (QoS), customization of user infrastructure, cost effectiveness, and simplified access interfaces, almost all of which are enabled by virtualization.  
Cloud infrastructure commonly requires large-scale data centers with high speed networking to support large numbers of systems.  
High Performance Computing (HPC) shares many of the infrastructure requirements and capabilities of these large computing centers so may also benefit from innovations originating in the cloud and virtual systems \autocite{xavier2013performance, _younge_1}.

Virtual machines have been the common case for systems virtualization, with containers taking a secondary role as a process isolation mechanism.
Network Functions Virtualization (NFV) and cloud computing have also grown recently, driving performance improvements of virtual systems, including both virtual machines and containers.
In NFV, the functionality that used to be embodied in physical devices such as routers, firewalls, and load balancers can now be realized within a virtual system which increases overall network flexibility and scalability in many cases.
A number of trade organizations have sprouted up recently to promote its adoption and accelerate development.
Both the European Telecommunications Standards Institute (ETSI) and Open Platform for Network Functions Virtualization (OPNFV) \autocite{opnfv1} are driving NFV adoption \autocite{cohnopnfv}.
In their survey of network virtualization, Chowdhury and Boutaba describe some of the older technologies and challenges that define this problem space~\autocite{_chowdhury_1}.

In addition to growth in virtualization technologies and uses, the need for high-throughput systems with deterministic responses is also expanding.
The Linux Foundation recently announced that they are fully supporting the efforts to promote and advance the development of realtime Linux \autocite{_linux_foundation_1}.
The Linux Foundation is the body that guides and sponsors the direction of Linux so their interest comes at an important time for realtime Linux due to recent funding difficulties for realtime Linux \autocite{_lwn_1} so their interest in realtime Linux provide a critical boost to its development and adoption.
Virtual systems also benefit from developments to realtime Linux due to improvements in system determinism and responsiveness.
% section introusecasesvt (end)

\section{Performance Analysis of Virtual Systems} % (fold)
\label{sec:introperformanceanalysis}
The use cases for virtualization in network-resident workloads are numerous, but performance of virtual systems is often a concern.
Architectural details of virtual machines and containers are discussed in Sections~\ref{sec:vmarchitecture} and \ref{sec:containerarchitecture}, respectively.  
Innovations to virtual machine and container architecture are discussed with an emphasis on their relevance to improving performance.

As Section~\ref{sec:performanceanalysis} shows, performance comparisons of containers and virtual machines seldom address the idea that these systems can be adjusted to fit various roles, depending on the level of resources available to them.  
Tuning of the kernel and process isolation can improve performance of multiple subsystems, here, network throughput and latency.  
% section introperformanceanalysis (end)


This thesis is divided into two parts.
Part one, composed of Chapters~\ref{sec:related_work}, and \ref{sec:experimental_design} describes some of the concerns involved in virtual systems and the work that has been done to improve their performance.
Part two, comprising Chapters~\ref{sec:results} and \ref{sec:conclusions} describe the performance results of the comparison performed here, conclusions that may be drawn from them, and the potential for further analysis.
